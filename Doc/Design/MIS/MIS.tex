\documentclass[12pt, titlepage]{article}

\usepackage{booktabs}
\usepackage{tabularx}
\usepackage{hyperref}
\usepackage{graphicx}
\usepackage{paralist}
\usepackage{xspace}
\usepackage{amsfonts}
\usepackage{amsmath}
\usepackage[normalem]{ulem}
\hypersetup{
    colorlinks,
    citecolor=black,
    filecolor=black,
    linkcolor=blue,
    urlcolor=blue
}
\usepackage[round]{natbib}

\title{SE 3XA3: Module Interface Specification\\Save The Date}

\author{
        Karuka Khurana (khurak1)\\
        Utsharga Rozario (rozariou)\\
        Samarth Kumar (kumars38)\\
        Dhruv Cheemakurti (cheemakd)\\
        \\
}

\date{\today}

\begin{document}

\maketitle

\pagenumbering{roman}
\tableofcontents
\listoftables
\listoffigures

\newpage

\begin{table}[h!]
\caption{\bf Revision History}
\begin{tabularx}{\textwidth}{p{3cm}p{2cm}X}
\toprule {\bf Date} & {\bf Version} & {\bf Notes}\\
\midrule
March 14 & 1.0 &
\begin{tabular}{>{\raggedright\arraybackslash}X}Initial React Doc By Utsharga\end{tabular}\\
\midrule
March 16 & 1.0 &
\begin{tabular}{>{\raggedright\arraybackslash}X}Updated React Doc By Utsharga\\Updated Python Doc By Samarth\end{tabular}\\
\midrule
March 18 & 1.1 &
\begin{tabular}{>{\raggedright\arraybackslash}X}Updated React Doc By Utsharga\\Updated Python Doc By Samarth\end{tabular}\\
\bottomrule
\end{tabularx}
\end{table}

\clearpage

\pagenumbering{arabic}

\newpage

\section{editablePage Module}

\subsection{Template Module extends React.Component}

editablePage

\subsection{Uses}

editableBlock, uid, caretHelpers

\subsection{Description}
This module represents the page. It details the suite of functions that are able to be executed on a page: editing, adding and deleting. It is therefore a controller module.

\subsection{Syntax}

\subsubsection{Exported Constants}

$\mathit{initialBlock}: \text{EditableBlock}$

\subsubsection{Exported Types}

ReactDOM

\subsubsection{Exported Access Programs}

\begin{tabular}{| l | l | l | l |}
  \hline
  \textbf{Routine name} & \textbf{In} & \textbf{Out} & \textbf{Exceptions}\\
  \hline
  constructor & HTML attribute & ~ & ~\\
  \hline
  updatePageHandler & EditableBlock & ~ & ~\\
  \hline
  addBlockHandler & EditableBlock & ~ & ~\\
  \hline
  deleteBlockHandler & EditableBlock & ~ & ~\\
  \hline
  render & ~ & ReactDOM & ~\\
  \hline
\end{tabular}

\subsection{Semantics}

\subsubsection{State Variables}

$\mathit{props}: \text{HTML attribute}$\\
$\mathit{updatedBlock}: \text{EditableBlock}$\\
$\mathit{currentBlock}: \text{EditableBlock}$

\subsubsection{State Invariant}

None

\subsubsection{Assumptions}

None.

\subsubsection{Access Routine Semantics}

\noindent constructor($props$):
\begin{itemize}
\item transition: $\mathit{state} = initialBlock$
\item output: $out := \mbox{self}$
\item exception: none
\end{itemize}

\noindent updatePageHandler(updatedBlock):
\begin{itemize}
\item transition: This function updates an editable block if the user makes any changes to the block.
\item exception: none
\end{itemize}

\noindent addBlockHandler(currentBlock):
\begin{itemize}
\item transition: This function adds addition blocks if the user enters a new block.
\item exception: none
\end{itemize}

\noindent deleteBlockHandler(currentBlock):
\begin{itemize}
\item transition: This fucntion deletes the block the user choose to delete.
\item exception: none
\end{itemize}

\noindent render():
\begin{itemize}
\item output: This function renders the output of the editable page.
\item exception: none
\end{itemize}

\newpage

\section{editableBlock Module}

\subsection{Template Module extends React.Component}

editableBlock

\subsection{Uses}

selectMenu, caretHelpers

\subsection{Description}
This module represents a block in the page. It details the suite of functions that are able to be executed on a block: editing, adding and deleting. It is therefore a controller module.

\subsection{Syntax}

\subsubsection{Exported Constants}

$\mathit{CMD_KEY}: \text{Character}$

\subsubsection{Exported Types}

ReactDOM

\subsubsection{Exported Access Programs}

\begin{tabular}{| l | l | l | l |}
  \hline
  \textbf{Routine name} & \textbf{In} & \textbf{Out} & \textbf{Exceptions}\\
  \hline
  constructor & HTML attribute & ~ & ~\\
  \hline
  componentDidMount & ~ & ~ & ~\\
  \hline
  componentDidUpdate & EditableTable.state & ~ & ~\\
  \hline
  onChangeHandler & keystroke & ~ & ~\\
  \hline
  onKeyDownHandler & keystroke & ~ & ~\\
  \hline
  onKeyUpHandler & keystroke & ~ & ~\\
  \hline
  openSelectMenuHandler & ~ & ~ & ~\\
  \hline
  closeSelectMenuHandler & ~ & ~ & ~\\
  \hline
  tagSelectionHandler & String & ~ & ~\\
  \hline
  render & ~ & ReactDOM & ~\\
  \hline
  
\end{tabular}

\subsection{Semantics}

\subsubsection{State Variables}

$\mathit{props}: \text{HTML attribute}$\\
$\mathit{prevState}: \text{EditableTable.state}$

\subsubsection{State Invariant}

$\mathit{e}: \text{keystroke}$

\subsubsection{Assumptions}

None.

\subsubsection{Access Routine Semantics}

\noindent constructor($props$):
\begin{itemize}
\item transition: $\mathit{state} = null $
\item output: $out := \mbox{self}$
\item exception: none
\end{itemize}

\noindent componentDidMount():
\begin{itemize}
\item transition: This function sets the current state to the object state.
\item exception: none
\end{itemize}

\noindent componentDidUpdate(prevState):
\begin{itemize}
\item transition: This function updates the page components if the user has either changed the html content or the tag.
\item exception: none
\end{itemize}

\noindent onChangeHandler(e):
\begin{itemize}
\item transition: This function handles any changes to the HTML content of an EditableBlock.
\item exception: none
\end{itemize}

\noindent onKeyDownHandler(e):
\begin{itemize}
\item transition: This function handles any changes to the block from the user using key strokes.
\item exception: none
\end{itemize}

\noindent onKeyUpHandler(e):
\begin{itemize}
\item transition: This function handles the SelectMenu on KeyUp input from the user.
\item exception: none
\end{itemize}

\noindent openSelectMenuHandler():
\begin{itemize}
\item transition: This function operates after opening the SelectMenu, it then attaches a click listener to the dom.
\item output: This function closes the menu after the next click - regardless of outside or inside menu.
\item exception: none
\end{itemize}

\noindent closeSelectMenuHandler():
\begin{itemize}
\item transition: This function changes state of the SelectMenu. 
\item output: This function closes the SelectMenu.
\item exception: none
\end{itemize}

\noindent tagSelectionHandler(tag):
\begin{itemize}
\item transition: This function assigns the tag to the EditableBlock.
\item exception: none
\end{itemize}

\noindent render():
\begin{itemize}
\item output: This function renders the output of the EditableBlock and SelectMenu
\item exception: none
\end{itemize}

\newpage

\section{selectMenu Module}

\subsection{Template Module extends React.Component}

selectMenu

\subsection{Uses}

None

\subsection{Description}
This module represents a menu in the page. It details the suite of functions that are able to be executed on a menu: selecting. It is therefore a controller module.

\subsection{Syntax}

\subsubsection{Exported Constants}

$\mathit{allowedTags}: \text{Array of String}$\\
$\mathit{MENU\_HEIGHT}: \text{Number}$

\subsubsection{Exported Types}

ReactDOM

\subsubsection{Exported Access Programs}

\begin{tabular}{| l | l | l | l |}
  \hline
  \textbf{Routine name} & \textbf{In} & \textbf{Out} & \textbf{Exceptions}\\
  \hline
  constructor & HTML attribute & ~ & ~\\
  \hline
  componentDidMount & ~ & ~ & ~\\
  \hline
  componentDidUpdate & selectMenu.state & ~ & ~\\
  \hline
  componentWillUnmount & ~ & ~ & ~\\
  \hline
  keyDownHandler & keystroke & ~ & ~\\
  \hline
  render & ~ & ReactDOM & ~\\
  \hline
\end{tabular}

\subsection{Semantics}

\subsubsection{State Variables}

$\mathit{props}: \text{HTML attribute}$\\
$\mathit{updatedBlock}: \text{EditableBlock}$\\
$\mathit{currentBlock}: \text{EditableBlock}$

\subsubsection{State Invariant}

$\mathit{e}: \text{keystroke}$

\subsubsection{Assumptions}

None.

\subsubsection{Access Routine Semantics}

\noindent constructor($props$):
\begin{itemize}
\item transition: $\mathit{state} = null $
\item output: $out := \mbox{self}$
\item exception: none
\end{itemize}

\noindent componentDidMount():
\begin{itemize}
\item transition: This function attaches a key listener to add any given key to the command
\item exception: none
\end{itemize}

\noindent componentDidUpdate(prevState):
\begin{itemize}
\item transition: This function checks whenever the command changes and looks for matching tags in the allowed list.
\item exception: none
\end{itemize}

\noindent componentWillMount():
\begin{itemize}
\item transition: This function unmounts the key listener.
\item exception: none
\end{itemize}

\noindent onKeyDownHandler(e):
\begin{itemize}
\item transition: This function hands the user input through key strokes.
\item exception: none
\end{itemize}

\noindent render():
\begin{itemize}
\item output: This function renders the output of the select menu option
\item exception: none
\end{itemize}

\newpage

\section{button Module}

\subsection{Template Module extends React.Component}

button

\subsection{Uses}

EditableBlock, EditableTable, loadPDF, scrapePDF, image, SelectMenu

\subsection{Description}
This module represents the button on the side of every block. It details the suite of functions that are able to be executed on the block. It is therefore a controller module.

\subsection{Syntax}

\subsubsection{Exported Constants}

None.

\subsubsection{Exported Types}

ReactDOM

\subsubsection{Exported Access Programs}

\begin{tabular}{| l | l | l | l |}
  \hline
  \textbf{Routine name} & \textbf{In} & \textbf{Out} & \textbf{Exceptions}\\
  \hline
  constructor & HTML attribute & ~ & ~\\
  \hline
  OptionHander & keystroke & ~ & ~\\
  \hline
  render & ~ & ReactDOM & ~\\
  \hline
\end{tabular}

\subsection{Semantics}

\subsubsection{State Variables}

None.

\subsubsection{State Invariant}

None.

\subsubsection{Assumptions}

$\mathit{e}: \text{keystroke}$

\subsubsection{Access Routine Semantics}

\noindent constructor($props$):
\begin{itemize}
\item transition: $\mathit{state} = initialBlock$
\item output: $out := \mbox{self}$
\item exception: none
\end{itemize}

\noindent OptionHander(e):
\begin{itemize}
\item output: This function obtains the user input as one of the available options which it then initiates.
\item exception: none
\end{itemize}

\noindent render():
\begin{itemize}
\item output: This function renders the button and the selectMenu on click.
\item exception: none
\end{itemize}

\newpage

\section{editableTable Module}

\subsection{Template Module extends React.Component}

editableTable

\subsection{Uses}

Table, loadData, utils, colors

\subsection{Description}
This module represents a table in the page. It details the suite of functions that are able to be executed on a menu: adding, editing and deleting. It is therefore a controller module.

\subsection{Syntax}

\subsubsection{Exported Constants}

None.

\subsubsection{Exported Types}

ReactDOM

\subsubsection{Exported Access Programs}

\begin{tabular}{| l | l | l | l |}
  \hline
  \textbf{Routine name} & \textbf{In} & \textbf{Out} & \textbf{Exceptions}\\
  \hline
  constructor & HTML attribute & ~ & ~\\
  \hline
  reducer & Table.state, Table.action & ~ & ~\\
  \hline
  render & ~ & ReactDOM & ~\\
  \hline
\end{tabular}

\subsection{Semantics}

\subsubsection{State Variables}

$\mathit{props}: \text{HTML attribute}$\\
$\mathit{state}: \text{Table.state}$\\
$\mathit{action}: \text{Table.action}$

\subsubsection{State Invariant}

None

\subsubsection{Assumptions}

None.

\subsubsection{Access Routine Semantics}

\noindent constructor($props$):
\begin{itemize}
\item transition: $\mathit{state} = null $
\item output: $out := \mbox{self}$
\item exception: none
\end{itemize}

\noindent reducer(state, action):
\begin{itemize}
\item transition: This function is is used to reduce the actions performed on the EditableTable.
\item exception: none
\end{itemize}

\noindent render():
\begin{itemize}
\item output: This function returns the rendering of the EditableTable.
\item exception: none
\end{itemize}

\newpage

\section{Table Module}

\subsection{Template Module extends React.Component}

Table

\subsection{Uses}

Cell, Header

\subsection{Description}
This module represents a table in the page. Its function is to only render the table itself. It is therefore a boundary module.

\subsection{Syntax}

\subsubsection{Exported Constants}

$\mathit{defaultColumn}: \text{Array of String}$

\subsubsection{Exported Types}

ReactDOM

\subsubsection{Exported Access Programs}

\begin{tabular}{| l | l | l | l |}
  \hline
  \textbf{Routine name} & \textbf{In} & \textbf{Out} & \textbf{Exceptions}\\
  \hline
  constructor & 2D Array, Array & ~ & ~\\
  \hline
  useMemo & ~ & Number & ~\\
  \hline
  useTable & ~ & ~ & ~\\
  \hline
  isTableResizing & ~ & Boolean & ~\\
  \hline
  render & ~ & ReactDOM & ~\\
  \hline
\end{tabular}

\subsection{Semantics}

\subsubsection{State Variables}

$\mathit{data}: \text{2D Array}$\\
$\mathit{columns}: \text{Array}$

\subsubsection{State Invariant}

None

\subsubsection{Assumptions}

None.

\subsubsection{Access Routine Semantics}

\noindent constructor($props$):
\begin{itemize}
\item transition: $\mathit{state} = data, columns $
\item output: $out := \mbox{self}$
\item exception: none
\end{itemize}

\noindent useMemo():
\begin{itemize}
\item output: This function sets up all of the rows and columns.
\item exception: none
\end{itemize}

\noindent useTable():
\begin{itemize}
\item transition: This function adjusts all of the sizing of the table.
\item exception: none
\end{itemize}

\noindent isTableResizing():
\begin{itemize}
\item output: This function checks if the table needs resizing.
\item exception: none
\end{itemize}

\noindent render():
\begin{itemize}
\item output: This function renders the outcome of the Table.
\item exception: none
\end{itemize}

\newpage

\section{Header Module}

\subsection{Template Module extends React.Component}

Header

\subsection{Uses}

utils, Text, colors

\subsection{Description}
This module represents a Header of a Table in the page. It details the suite of functions that are able to be executed on a menu: selecting. It is therefore a controller module.

\subsection{Syntax}

\subsubsection{Exported Constants}

$\mathit{buttons}: \text{Array of Strings}$\\
$\mathit{types}: \text{Array of Strings}$\\

\subsubsection{Exported Types}

ReactDOM

\subsubsection{Exported Access Programs}

\begin{tabular}{| l | l | l | l |}
  \hline
  \textbf{Routine name} & \textbf{In} & \textbf{Out} & \textbf{Exceptions}\\
  \hline
  constructor & HTML attribute & ~ & ~\\
  \hline
  setExpanded & Boolean & ~ & ~\\
  \hline
  setHeader & String & ~ & ~\\
  \hline
  handleKeyDown & keystroke & ~ & ~\\
  \hline
  handleChange & keystroke & ~ & ~\\
  \hline
  handleBlur & keystroke & ~ & ~\\
  \hline
  render & ~ & ReactDOM & ~\\
  \hline
\end{tabular}

\subsection{Semantics}

\subsubsection{State Variables}

$\mathit{props}: \text{HTML attribute}$\\
$\mathit{label}: \text{String}$

\subsubsection{State Invariant}

$\mathit{e}: \text{keystroke}$

\subsubsection{Assumptions}

None.

\subsubsection{Access Routine Semantics}

\noindent constructor($props$):
\begin{itemize}
\item transition: $\mathit{state} = null$
\item output: $out := \mbox{self}$
\item exception: none
\end{itemize}

\noindent setExpanded():
\begin{itemize}
\item transition: This function sets up the Header to be expandable.
\item exception: none
\end{itemize}

\noindent setHeader(label):
\begin{itemize}
\item transition: This function sets the Header to the label.
\item exception: none
\end{itemize}

\noindent handleKeyDown():
\begin{itemize}
\item transition: This function handles the user input on key down strokes.
\item exception: none
\end{itemize}

\noindent handleChange(e):
\begin{itemize}
\item transition: This function handles changes in the user input through key strokes.
\item exception: none
\end{itemize}

\noindent handleBlur(e):
\begin{itemize}
\item transition: This function updates the Header.
\item exception: none
\end{itemize}

\noindent render():
\begin{itemize}
\item output: This function renders the outcome of the Header.
\item exception: none
\end{itemize}

\newpage

\section{Cell Module}

\subsection{Template Module extends React.Component}

Cell

\subsection{Uses}

utils, colors

\subsection{Description}
This module represents a Cell in a Table in the page. It details the suite of functions that are able to be executed on a menu: selecting, editing and deleting. It is therefore a controller module.

\subsection{Syntax}

\subsubsection{Exported Constants}

None

\subsubsection{Exported Types}

ReactDOM

\subsubsection{Exported Access Programs}

\begin{tabular}{| l | l | l | l |}
  \hline
  \textbf{Routine name} & \textbf{In} & \textbf{Out} & \textbf{Exceptions}\\
  \hline
  constructor & HTML attribute & ~ & ~\\
  \hline
  setValue & String & ~ & ~\\
  \hline
  dataDispatch & Cell & ~ & ~\\
  \hline
  handleOptionKeyDown & keystroke & ~ & ~\\
  \hline
  handleAddOption & keystroke & ~ & ~\\
  \hline
  handleOptionBlur & keystroke & ~ & ~\\
  \hline
  render & ~ & ReactDOM & ~\\
  \hline
\end{tabular}

\subsection{Semantics}

\subsubsection{State Variables}

$\mathit{props}: \text{HTML attribute}$\\
$\mathit{value}: \text{string}$\\
$\mathit{dataCell}: \text{Cell}$

\subsubsection{State Invariant}

$\mathit{e}: \text{keystroke}$

\subsubsection{Assumptions}

None.

\subsubsection{Access Routine Semantics}

\noindent constructor($props$):
\begin{itemize}
\item transition: $\mathit{state} = null $
\item output: $out := \mbox{self}$
\item exception: none
\end{itemize}

\noindent setValue(value):
\begin{itemize}
\item transition: This function set the value based on input.
\item exception: none
\end{itemize}

\noindent dataDispatch(dataCell):
\begin{itemize}
\item transition: This function updates data using dataCell.
\item exception: none
\end{itemize}

\noindent handleOptionKeyDown(e):
\begin{itemize}
\item transition: This function handles KeyDown options from user input.
\item exception: none
\end{itemize}

\noindent handleAddOption(e):
\begin{itemize}
\item transition: This function handles additions rows added to the column.
\item exception: none
\end{itemize}

\noindent handleOptionBlur(e):
\begin{itemize}
\item transition: This function updates data based on the key stroke.
\item exception: none
\end{itemize}

\noindent render():
\begin{itemize}
\item output: This function renders the output of the Cell.
\item exception: none
\end{itemize}

\newpage

\section{Relationship Module}

\subsection{Template Module extends React.Component}

Relationship

\subsection{Uses}

None

\subsection{Description}
This module represents a Relationship in the Cell of a Table in the page. It function to only render the relationship. It is therefore a boundary module.

\subsection{Syntax}

\subsubsection{Exported Constants}

colors

\subsubsection{Exported Types}

ReactDOM

\subsubsection{Exported Access Programs}

\begin{tabular}{| l | l | l | l |}
  \hline
  \textbf{Routine name} & \textbf{In} & \textbf{Out} & \textbf{Exceptions}\\
  \hline
  render & ~ & ReactDOM & ~\\
  \hline
\end{tabular}

\subsection{Semantics}

\subsubsection{State Variables}

None.

\subsubsection{State Invariant}

None.

\subsubsection{Assumptions}

None.

\subsubsection{Access Routine Semantics}

\noindent render():
\begin{itemize}
\item output: $ReactDOM$
\item exception: none
\end{itemize}

\section{Text Module}

\subsection{Template Module extends React.Component}

Text

\subsection{Uses}

None

\subsection{Description}
This module represents a Text in the Cell of a Table in the page. It function to only return a color as String. It is therefore a entity module.

\subsection{Syntax}

\subsubsection{Exported Constants}

None

\subsubsection{Exported Types}

ReactDOM

\subsubsection{Exported Access Programs}

\begin{tabular}{| l | l | l | l |}
  \hline
  \textbf{Routine name} & \textbf{In} & \textbf{Out} & \textbf{Exceptions}\\
  \hline
  render & ~ & ReactDOM & ~\\
  \hline
\end{tabular}

\subsection{Semantics}

\subsubsection{State Variables}

None.

\subsubsection{State Invariant}

None.

\subsubsection{Assumptions}

None.

\subsubsection{Access Routine Semantics}

\noindent render():
\begin{itemize}
\item output: $ReactDOM$
\item exception: none
\end{itemize}

\section{Colors Module}

\subsection{Template Module extends React.Component}

Colors

\subsection{Uses}

None

\subsection{Description}
This module represents a colors in the Cell of a Table in the page. It function to only render the relationship. It is therefore a entity module.

\subsection{Syntax}

\subsubsection{Exported Constants}

None

\subsubsection{Exported Types}

ReactDOM

\subsubsection{Exported Access Programs}

\begin{tabular}{| l | l | l | l |}
  \hline
  \textbf{Routine name} & \textbf{In} & \textbf{Out} & \textbf{Exceptions}\\
  \hline
  grey & Number & Number & ~\\
  \hline
\end{tabular}

\subsection{Semantics}

\subsubsection{State Variables}

$\mathit{value}: \text{Number}$

\subsubsection{State Invariant}

None.

\subsubsection{Assumptions}

None.

\subsubsection{Access Routine Semantics}

\noindent grey(value):
\begin{itemize}
\item output: $out =$ String
\item exception: none
\end{itemize}

\newpage

\section{uid Module}

\subsection{Template Module}

uid

\subsection{Uses}

None

\subsection{Description}
This module represents a uid generated for EditablePage and EditableBlock. It function to only output a string to other functions. It is therefore a entity module.

\subsection{Syntax}

\subsubsection{Exported Constants}

None

\subsubsection{Exported Types}

ReactDOM

\subsubsection{Exported Access Programs}

\begin{tabular}{| l | l | l | l |}
  \hline
  \textbf{Routine name} & \textbf{In} & \textbf{Out} & \textbf{Exceptions}\\
  \hline
  uid & ~ & String & ~\\
  \hline
\end{tabular}

\subsection{Semantics}

\subsubsection{State Variables}

None.

\subsubsection{State Invariant}

None.

\subsubsection{Assumptions}

None.

\subsubsection{Access Routine Semantics}

\noindent uid():
\begin{itemize}
\item output: $out =$ String
\item exception: none
\end{itemize}

\section{caretHelpers Module}

\subsection{Template Module}

caretHelpers

\subsection{Uses}

None

\subsection{Description}
This module gets and sets coordinates for EditableBlock. It is therefore a controller module.

\subsection{Syntax}

\subsubsection{Exported Constants}

None

\subsubsection{Exported Types}

ReactDOM

\subsubsection{Exported Access Programs}

\begin{tabular}{| l | l | l | l |}
  \hline
  \textbf{Routine name} & \textbf{In} & \textbf{Out} & \textbf{Exceptions}\\
  \hline
  getCaretCoordinates & ~ & Number, Number & ~\\
  \hline
  setCaretToEnd & Object & ~ & ~\\
  \hline
\end{tabular}

\subsection{Semantics}

\subsubsection{State Variables}

$\mathit{element}: \text{Object}$\\

\subsubsection{State Invariant}

None.

\subsubsection{Assumptions}

None.

\subsubsection{Access Routine Semantics}

\noindent getCaretCoordinates():
\begin{itemize}
\item output: $out =$ Number, Number
\item exception: none
\end{itemize}

\noindent setCaretToEnd(element):
\begin{itemize}
\item transition: This function sets the caret to the end of the existing components.
\item exception: none
\end{itemize}

\newpage

\section{utils Module}

\subsection{Template Module}

utils

\subsection{Uses}

None

\subsection{Description}
This module generates short IDs and random colours. It is therefore a entity module.

\subsection{Syntax}

\subsubsection{Exported Constants}

None

\subsubsection{Exported Types}

ReactDOM

\subsubsection{Exported Access Programs}

\begin{tabular}{| l | l | l | l |}
  \hline
  \textbf{Routine name} & \textbf{In} & \textbf{Out} & \textbf{Exceptions}\\
  \hline
  shortId & ~ & String & ~\\
  \hline
  randomColor & ~ & String & ~\\
  \hline
\end{tabular}

\subsection{Semantics}

\subsubsection{State Variables}

None.

\subsubsection{State Invariant}

None.

\subsubsection{Assumptions}

None.

\subsubsection{Access Routine Semantics}

\noindent shortId():
\begin{itemize}
\item output: This function creates a short unique ID from strings.
\item exception: none
\end{itemize}

\noindent randomColor():
\begin{itemize}
\item output: This function generates random colours.
\item exception: none
\end{itemize}

\newpage

\section{loadPDF Module}

\subsection{Template Module extends React.Component}

loadPDF

\subsection{Uses}

None.

\subsection{Syntax}

\subsubsection{Exported Constants}

None.

\subsubsection{Exported Types}

ReactDOM

\subsubsection{Exported Access Programs}

\begin{tabular}{| l | l | l | l |}
  \hline
  \textbf{Routine name} & \textbf{In} & \textbf{Out} & \textbf{Exceptions}\\
  \hline
  constructor & HTML attribute & ~ & ~\\
  \hline
  onDocumentLoadSuccess & Number & ~ & ~\\
  \hline
  changePage & Number & ~ & ~\\
  \hline
  previousPage & ~ & ~ & ~\\
  \hline
  nextPage & ~ & ~ & ~\\
  \hline
  render & ~ & ReactDOM & ~\\
  \hline
\end{tabular}

\subsection{Semantics}

\subsubsection{State Variables}

$\mathit{props}: \text{HTML attribute}$\\
$\mathit{numPages}: \text{Number}$\\
$\mathit{offset}: \text{Number}$

\subsubsection{State Invariant}

None

\subsubsection{Assumptions}

None.

\subsubsection{Access Routine Semantics}

\noindent constructor($props$):
\begin{itemize}
\item transition: $\mathit{state} = initialBlock$
\item output: $out := \mbox{self}$
\item exception: none
\end{itemize}

\noindent onDocumentLoadSuccess(numPages):
\begin{itemize}
\item transition: This function sets the number of pages and the initial page number to 1
\item exception: none
\end{itemize}

\noindent changePage(offset):
\begin{itemize}
\item transition: This function calcuates the current page.
\item exception: none
\end{itemize}

\noindent previousPage():
\begin{itemize}
\item transition: This function calculates the previous page by decrementing the current page.
\item exception: none
\end{itemize}

\noindent nextPage():
\begin{itemize}
\item transition: This function calculates the next page by incrementing the current page.
\item exception: none
\end{itemize}

\noindent render():
\begin{itemize}
\item output: This function renders the output of the PDF.
\item exception: none
\end{itemize}

\newpage

\section{scrapePDFWindow Module}

\subsection{Template Module extends React.Component}

scrapePDFWindow

\subsection{Uses}

loadData, EditableTable

\subsection{Description}
This module represents the pop-up window when the user wants to scrape a document. It details the suite of functions that are able to be executed on the window: adding course name and selecting the PDF. It is therefore a controller module.

\subsection{Syntax}

\subsubsection{Exported Constants}

$\mathit{courseName}: \text{String}$\\
$\mathit{pageStart}: \text{Number}$\\
$\mathit{pageEnd}: \text{Number}$

\subsubsection{Exported Types}

ReactDOM

\subsubsection{Exported Access Programs}

\begin{tabular}{| l | l | l | l |}
  \hline
  \textbf{Routine name} & \textbf{In} & \textbf{Out} & \textbf{Exceptions}\\
  \hline
  constructor & HTML attribute & ~ & ~\\
  \hline
  getCourseName & ~ & String & ~\\
  \hline
  getPageStart & ~ & Number & ~\\
  \hline
  getPageEnd & ~ & Number & ~\\
  \hline
  getFile & ~ & ~ & ~\\
  \hline
  scrape & ~ & ~ & ~\\
  \hline
  render & ~ & ReactDOM & ~\\
  \hline
\end{tabular}

\subsection{Semantics}

\subsubsection{State Variables}

$\mathit{props}: \text{HTML attribute}$\\
$\mathit{numPages}: \text{Number}$\\
$\mathit{offset}: \text{Number}$

\subsubsection{State Invariant}

None

\subsubsection{Assumptions}

None.

\subsubsection{Access Routine Semantics}

\noindent constructor($props$):
\begin{itemize}
\item transition: $\mathit{state} = initialBlock$
\item output: $out := \mbox{self}$
\item exception: none
\end{itemize}

\noindent getCourseName():
\begin{itemize}
\item output: This function obtains the Course Name from the User
\item exception: none
\end{itemize}

\noindent getPageStart():
\begin{itemize}
\item output: This function obtains the starting page to scrap from the User
\item exception: none
\end{itemize}

\noindent getPageEnd():
\begin{itemize}
\item output: This function obtains the end page to scrape from the User
\item exception: none
\end{itemize}

\noindent getFile():
\begin{itemize}
\item transition: This function obtains the file to scrape from the User.
\item exception: none
\end{itemize}

\noindent scrape(fileName):
\begin{itemize}
\item transition: This function uses the user inputs from getCourseName, getPageStart, getPageEnd and getFile to initiate the python scraping.
\item output: This function returns a 2D array of table that was obtained from scraping.
\item exception: none
\end{itemize}

\noindent render():
\begin{itemize}
\item output: This function renders the output of the scraping as a table.
\item exception: none
\end{itemize}

\newpage

\section{image Module}

\subsection{Template Module extends React.Component}

image

\subsection{Uses}

None.

\subsection{Description}
This module represents an image on the page. It details the suite of functions that are able to be executed on the image: adding the image and rendering it. It is therefore a controller module.

\subsection{Syntax}

\subsubsection{Exported Constants}

None.

\subsubsection{Exported Types}

ReactDOM

\subsubsection{Exported Access Programs}

\begin{tabular}{| l | l | l | l |}
  \hline
  \textbf{Routine name} & \textbf{In} & \textbf{Out} & \textbf{Exceptions}\\
  \hline
  constructor & HTML attribute & ~ & ~\\
  \hline
  getPictureLocation & ~ & ~ & ~\\
  \hline
  render & ~ & ReactDOM & ~\\
  \hline
\end{tabular}

\subsection{Semantics}

\subsubsection{State Variables}

$\mathit{props}: \text{HTML attribute}$\\
$\mathit{numPages}: \text{Number}$\\
$\mathit{offset}: \text{Number}$

\subsubsection{State Invariant}

None

\subsubsection{Assumptions}

None.

\subsubsection{Access Routine Semantics}

\noindent constructor($props$):
\begin{itemize}
\item transition: $\mathit{state} = initialBlock$
\item output: $out := \mbox{self}$
\item exception: none
\end{itemize}

\noindent getImageLocation():
\begin{itemize}
\item output: This function obtains the image location from the User
\item exception: none
\end{itemize}

\noindent render():
\begin{itemize}
\item output: This function renders the output as an image component.
\item exception: none
\end{itemize}

\end{document}

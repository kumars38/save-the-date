\documentclass{article}

% Language setting
% Replace `english' with e.g. `spanish' to change the document language
\usepackage[english]{babel}

% Set page size and margins
% Replace `letterpaper' with`a4paper' for UK/EU standard size
\usepackage[letterpaper,top=1cm,bottom=2cm,left=2cm,right=2cm,marginparwidth=1cm]{geometry}

% Useful packages
\usepackage{amsmath}
\usepackage{graphicx}
\usepackage[colorlinks=true, allcolors=blue]{hyperref}

\title{3XA3 L03 Group 17\\Project Approval Document}
\author{Karuka Khurana, Utsharga Rozario, Samarth Kumar}

\begin{document}
\maketitle

\begin{itemize}
    \item[] \textbf{Team Name:} Group 17

    \item[] \textbf{Group members/Mac ID:}\\
        Karuka Khurana (khurak1)\\
        Utsharga Rozario (rozariou)\\
        Samarth Kumar (kumars38)
        
    \item[] \textbf{Original Project Name:} Notion PDF Scrapper
    
    \item[] \textbf{Software Purpose:} The purpose of this project is to implement a PDF scraping feature to an existing open source React Notion renderer. 

    \item[] \textbf{Software Scope:} The scope of this project is to design, develop, and implement a Python based PDF scraper that would scrape a document and extract important dates/deadlines. We will create a dashboard that lists all the important dates and categorize them based on courses a student is taking. 

    \item[] \textbf{URL for original project:}  https://github.com/NotionX/react-notion-x  

    \item[] \textbf{Any specialized hardware requirements?} No

    \item[] \textbf{Any software license required that McMaster does not own?} No 

    \item[] \textbf{Programming language:} React, TypeScript, Python 

    \item[] \textbf{Is programming language feasible for your team?} Yes 

    \item[] \textbf{Is the domain knowledge understandable within one term?} Yes 

    \item[] \textbf{Number of lines of code:} Approximately 10,000 

    \item[] \textbf{License:} https://github.com/NotionX/react-notion-x/blob/master/license (MIT License) 

    \item[] \textbf{License allows public redevelopment?} Yes 

    \item[] \textbf{Can you compile the original projects source code?} Yes 

    \item[] \textbf{What would be some test cases for the existing software?}
        \begin{itemize}
            \item Create sample Notion web pages and render them using the existing software.
            \item Create a table/list of dates in a sample Notion web page
        \end{itemize}
    
    \item[] \textbf{What changes you intend to make to the project?}\\
    Our group intends to integrate some minor features such as YouTube integration and on-hover preview to the pre-existing Notion React renderer. The YouTube integration feature would allow a user to paste a YouTube URL within a Notion page and the video could be viewed in a player. The on-hover feature would allow a user to paste any URL in a Notion page and once a user hovers over the link, a preview of the website will be displayed. A major feature we hope to implement is a PDF scraper. A user can upload a PDF document (e.g. course outline). The software will scan the document for important due dates and deadlines by identifying pre-defined key words. It will then take this information and create a dashboard for the user that will visualize all the dates in an organized, easy-to-read manner. 
    
\end{itemize}

\end{document}
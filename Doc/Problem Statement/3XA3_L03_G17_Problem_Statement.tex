\documentclass[12pt, titlepage]{article}

\usepackage{cite}
\usepackage{booktabs}
\usepackage{tabularx}
\usepackage{hyperref}
\usepackage{amssymb}
\usepackage{amstext}
\usepackage{amsthm}
\usepackage{amsmath}
\usepackage{enumerate}
\usepackage{fancyhdr}
\usepackage[margin=1in]{geometry}
\usepackage{graphicx}
\usepackage{extarrows}
\usepackage{setspace}
\usepackage{adjustbox}
\usepackage{hyperref}


\hypersetup{
    colorlinks,
    citecolor=black,
    filecolor=black,
    linkcolor=blue,
    urlcolor=blue
}
%\usepackage[round]{natbib}

\title{SFWRENG 3XA3 L03: Problem Statement
        \\[1ex] \large SaveTheDate\\ Team 17}

\author{
        Karuka Khurana (khurak1)\\
        Utsharga Rozario (rozariou)\\
        Samarth Kumar (kumars38)\\
        Dhruv Cheemakurti (cheemakd)\\
        \\
}


\begin{document}

\maketitle

\pagenumbering{roman}
\tableofcontents
\listoftables
\listoffigures

\begin{table}[!hbp]
    \caption{Revision History} \label{RevisionHistory}
    \begin{tabularx}{\textwidth}{llX}
        \toprule
            \textbf{Date} & \textbf{Developer(s)} & \textbf{Change}\\
        \midrule
            January 25, 2022 & Karuka, Utsharga, Samarth & Initial draft, Problem Statement \& Stakeholders\\
            January 26, 2022 & Utsharga, Dhruv & Evironment \& Stakeholders\\
        \bottomrule
    \end{tabularx}
\end{table}

\newpage

\pagenumbering{arabic}
\maketitle

\newpage
\section{Problem Statement}
\begin{itemize}
    \item[] 
    Notion is an all-in-one work space where users can write, plan, collaborate and organize their thoughts, projects and other information. Notion has a variety of features that allow users to take notes, add tasks, and even create custom page layouts. Alongside its many features, Notion gives users the ability to upload PDF documents onto a page, providing a built-in view of the document, rather than having to download the PDF every time to view it. Currently, many McMaster students skim through multiple course outlines throughout each semester to jot down or remind themselves of important dates. As a result, students are not only spending extra time reading dense pages of course outlines, but also run the risk of missing important dates and deadlines for different courses.\\ 

    With students taking multiple courses per semester, our team would like to automate the process of sifting through documents for due dates and deadlines. We plan to create easy-to-read dashboards for users to quickly access such information for each respective course they are taking.
\end{itemize}

\section{Stakeholders}
\quad \quad The stakeholders of the application are:
\begin{itemize}
  \item Product users:
  \begin{itemize}
      \item[] The product users are University Students who will be using the application. The product users have an influence on the requirements of the application and its overall development.
  \end{itemize}
  \item Investors:
  \begin{itemize}
      \item[] Investors finance the development of the application by providing capital. Investors are directly affected by the performance of the application and strive to increase the likeability and usage of the product.
  \end{itemize}
  \item Development Team:
  \begin{itemize}
      \item[] The development team is the group of people who are developing the application. They include, but are not limited to, the UI/UX designers and software developers who develop the application and ensure easy to use user experience.
  \end{itemize}
  \item {Marketers:}
  \begin{itemize}
      \item[] Marketers is the group of people who will advertise the application to the market. The additional features will make our product more appealing for public usage and for users already in the Notion ecosystem that require a PDF scrapper.
  \end{itemize}
\end{itemize}

\subsection{Importance of the Problem}
\begin{itemize}
    \item[]
    McMaster students need a quick and easy way to manage important dates and deadlines for their courses. These are often provided within course outline documents. However, these documents can be long and difficult to manually sift through to find relevant information. Furthermore, this task becomes even more extensive when searching for dates and deadlines across multiple courses. Our application allows for the automatic extraction of these important dates, saving students the time and effort of doing this process manually.
\end{itemize}

\subsection{Environment for the Application}
\begin{itemize}
    \item[]
    The environment of the application would be determined by what students wish to use throughout their semesters. This can include any personal or McMaster desktops. 
\end{itemize}

\end{document}